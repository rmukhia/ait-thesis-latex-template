\setlength{\footskip}{8mm}

\chapter{Introduction}

\textit{An autonomous vehicle should be aware of its environment to take the next decision. SLAM (Simultaneous Localization and Mapping) systems can localize and map the environment at the same time. In this paper, I focus on the implementation of a Monocular Visual Inertial SLAM system that provide pose and odometry of the vehicle as well as point clouds of the environment.}

\section{Overview}

Autonomous driving vehicles have three principal components: perception, decision and control, and vehicle platform manipulation \shortcite{sbehere16t}. Perception means the data provided by the sensor can be used to understand the environment, localize the robot, and build a world model of the environment.

Visual simultaneous localization and mapping (SLAM) systems provide localization and mapping at the same time. They use cameras, which are now a cheap consumer commodity, and can be a cost-effective solution for perception.

Monocular vSLAM requires only one camera, but it suffers from issues of map initialization and scale ambiguity. That is the size of the environment mapped will not be to scale, so the vehicle may not be able to function properly.

Integrating accelerometer and gyroscope measurements can help determine an approximate scale during map initialization and update the approximate pose offset between subsequent image frames during tracking.


\section{Problem Statement}

An autonomous vehicle in unknown terrain has to perceive the environment around it for safe operation. The vehicle can be fitted with an array of sensors such as ultrasonic, infrared, multiarray laser sensors, and RGB-D cameras. Systems with many sensors have integration complexities and may suffer from weight constraints, reduced battery lifetime, and high cost. Monocular SLAM provides a lower cost, lighter weight, simpler alternative. I pursue Monocular SLAM with IMU integration in this study, which will form the basis of the perception component in a futute extension to this work.


\section{Objectives}

The main objective of this study is to generate real time point clouds and robot poses by executing a monocular visual inertial SLAM algorithm using the low-cost commercially available Mobius Maxi action camera and a seperate low cost flight controller, the PixHawk 2.4.8, for IMU data. By fusing the output of this system with odometry data from GPS, a robust system can be built that maps outdoor environments, even when vision-based tracking is lost and the vehicle cannot be re-localized though the SLAM algorithm.

\section{Limitations and Scope}

The scope of this study is to build a visual inertial SLAM system, that can provide real time point clouds and robot pose information to other components in the perception/action pipeline of an autonomous robot.

The limitations of this study are:

\begin{itemize}
	\item The system does not presently run on a single board computer such as the Odroid XU4. 
	\item The fusion system, which would be able to combine GPS data along with SLAM results, is not implemented, and has been left as future work.
\end{itemize}

\section{Thesis Outline}

I organize the rest of this dissertation as follows.

In Chapter \ref{ch:literature-review}, I provide a literature review.

In Chapter \ref{ch:methodology}, I propose my methodology.

In Chapter \ref{ch:results}, I present experimental results.

Finally, in Chapter \ref{ch:conclusion}, I conclude the special study.

\FloatBarrier
